\documentclass{article}

\usepackage{amsmath}
\usepackage{url}

\newcommand{\newword}[1]{{\bf #1}}
\newtheorem{problem}{Problem}
\newcommand{\NP}{\ensuremath{\mathcal{NP}}}

\title{Using Cell Phone Keyboards is (\NP) Hard}
\author{P. Boothe}

\begin{document}
\maketitle

\begin{abstract}
Sending text messages on cell phones which only contain the keys 0-9 and
\# and * can be a painful experience.  We consider the problem of
designing an optimal mapping of numbers to sets of letters to act as an
alternative to the standard $\{2\to\{abc\}, 3\to\{def\}\ldots\}$.  Our
overall goal is to minimize the number of keys that must be typed.  We
prove that the problem is \NP hard, but describe several mappings which
improve the standard one.
\end{abstract}

Typing on a keyboard which has fewer keys than there are letter in
the alphabet can be a painful task.  There are a plethora of 
input schemes which attempt to make this task easier, but the one thing
they all have in common is that all of these input methods use the
standard mapping of numeric keys to alphabetic numbers of $\{2\to\{abc\},
         3\to\{def\}, 4\to\{ghi\}, 5\to\{jkl\}, 6\to\{mno\}, 7\to\{pqrs\},
         8\to\{tuv\}, 9\to\{wxyz\}\}$.  They use this mapping for historic
reasons, but, if we don't mind breaking some ``legacy'' applications of the
phone system such as ``1-800-FLOWERS'' and the like, then we might try
rearranging the numbers on the keys to make messages easier to type.

Before we get any farther, let us sketch the basic problem that we will keep
revising and revisiting throughout this paper.
\begin{problem}[{\sc MinimumKeystrokes}]~\\
{\sc Instance}: A set of letters corresponding to an alphabet $A$ $(|A| =
n)$, a number of keys $k$, an input method $IN$, and a set of
tuples of words and frequencies $W$.  The frequencies in $W$ are integers,
and the words are made up of solely of elements of $A$.  We will
treat $IN$ as a function which, given a partition of $A$ and a word
$w$, returns how many keystrokes are required to type $w$.~\\
        ~\\
{\sc Question}: What is the best partition of $A$ into $k$ sets, such that the
total number of keystrokes to type every word in $W$ its associated frequency
times is minimized?  Equivalently, what is the partition of $P$ of $A$ that
will minimize
$$\sum_{(w,f)\in W} f*IN(w,P)$$
\end{problem}

Over the course of this paper we consider three different real-world
schemes for $IN$ (basic typing, T9, and predictive T9), and for each
variant that is proven \NP-hard, we consider the restriction on $P$ that
requires that we keep the alphabet in alphabetical order.  In all cases
where it is computationally feasible, we provide results for the case of
the English language, on 8-key keyboards, using the British National
Corpus\cite{bnc}.  

\bibliographystyle{abbrv}
\bibliography{bibliography}
\end{document}
